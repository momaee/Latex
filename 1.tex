\documentclass[conference]{IEEEtran}
\IEEEoverridecommandlockouts
% The preceding line is only needed to identify funding in the first footnote. If that is unneeded, please comment it out.
\usepackage{cite}
\usepackage{amsmath,amssymb,amsfonts}
\usepackage{algorithmic}
\usepackage{graphicx}
\usepackage{textcomp}
\def\BibTeX{{\rm B\kern-.05em{\sc i\kern-.025em b}\kern-.08em
  T\kern-.1667em\lower.7ex\hbox{E}\kern-.125emX}}
\begin{document}

\title{Paper Title*}

\author{
  \IEEEauthorblockN{1\textsuperscript{st} Given Name Surname}
  \IEEEauthorblockA{\textit{dept. name of organization (of Aff.)} \\
  \textit{name of organization (of Aff.)}\\
  City, Country \\
  email address}
\and
  \IEEEauthorblockN{2\textsuperscript{nd} Given Name Surname}
  \IEEEauthorblockA{\textit{dept. name of organization (of Aff.)} \\
  \textit{name of organization (of Aff.)}\\
  City, Country \\
  email address}
}

\maketitle

\begin{abstract}

\end{abstract}

\begin{IEEEkeywords}

\end{IEEEkeywords}

  \section{Introduction}
  \section{System Model}
    % \cite{ref1}

    % \begin{enumerate}
      % \item
      % \item
    % \end{enumerate}
    % We consider that measuremnts of sensors for service $s$ are performed and sent to processing units by rate of $\sum_{i=1}^{N_I} r_{i,s}$ and $r_{i,s}$ is sampeling rate processed by IaaS provider $i$ for service $s$, hence we use $\log(\sum_{i=1}^{N_I} r_{i,s})$ as a quality of experience measurement for user of service $s$.
    % Let $p_i$ be the price that should be paid to IaaS provider $i$ for unit of processing power and $F_s$ be the processing power required for each sensor sample of user $s$.
    % Then $p_i r_{i,s} F_s$ is price paid to IaaS provider $i$ by user of service $s$ and total cost for user of service $s$ is $\sum_{i=1}^{N_I}{p_i r_{i,s} F_s}$.
    % We define utility function for user of service $s$ as the difference between quality of experience indicator and his total cost:
    \begin{equation}
      \varphi_s(R_s)=\log(\sum_{i=1}^{N_I} r_{i,s}) - \sum_{i=1}^{N_I}{p_i r_{i,s} F_s}
    \end{equation}
    % Where $R_s$ is $[r_{1,s}, \hdots, r_{N_I,s}]$ and $\lambda_{i,s}$s are descision variables for user $s$.
    % We assume that each user's serivce is characterized by its mean response time and it should be greater than $R_s^{max}$ to meet user's requirements.
    % Mean response time for service $s$ can be calculated as:
    \begin{equation}
      t_s = \frac{1}{\sum_{i=1}^{N_I} r_{i,s}}\sum_{i=1}^{N_I}{r_{s,i}(t^{sensors}_{i,s} + t_{i,s} + t^{actuators}_{i,s})}
    \end{equation}
    % Where $t^{sensors}_{i,s}$ and $t^{actuators}_{i,s}$ are network delays from sensors to IaaS provider $i$ and from IaaS provider $i$ to actuator and $t_{i,s}$ is mean response time for service $s$ at IaaS provider $i$.
    % From queueing theory we know that for a M/M/1 queue with arival rate $\lambda$ and service rate $\mu$, mean response time is $\frac{1}{\mu - \lambda}$.
    % If we assume that $C_i$ is processing capasity of IaaS provider $i$ and $u_{i,s}$ is utilizaion of IaaS provider $i$ by service $s$ then for service $s$ on IaaS provider $i$ arival rate is $r_{i,s} R_s$ and service rate is $u_{i,s} C_i$, So $t_{i,s}$ can be written as follow:
    \begin{equation}
      t_{i,s} = \frac{1}{u_{i,s} C_i - r_{i,s} F_s}
    \end{equation}
    % Which should be greater than $r_s^{min}$.
    % To prevent IaaS provider from being saturated for user of service $s$ we consider that $r_{s,i} F_s \le 0.9 u_{i,s} C_i$.
    % Therefore each user try to solve following optimization problem:
    \begin{subequations}
      \begin{align}
        &\max_{\Lambda_s} \varphi_s(R_s) \\
        &\text{subject to:} \nonumber \\
        &R_s^{min} \le \sum_{k=1}^{N_I}{r_{k,s}} \\
        &0 \le r_{j,s}, \forall j \in \{1, \hdots,N_I\} \\
        &r_{j,s} R_s \le 0.9 u_{j,s} C_j, \forall j \in \{1, \hdots,N_I\} \\
        &t_{j,s} +  \frac{1}{u_{j,s} C_j - r_{j,s} F_s} \le 0
      \end{align}
    \end{subequations}
    % \par
    % We consider two types of IaaS providers:sssssssssssssssssssssssssssss
    % \begin{enumerate}
    %   \item Usual servers placed in datacenters which has vast amount of computational power.
    %   \item Wireless sensor gateways placed near sensors which has limited computational power, but may have less transmition delay.
    % \end{enumerate}
    % IaaS providers are paid for their processing power so their revenue is sum of all revenue come from services that has non-zero $\lambda_{i,s}$ and they pay for their electric power consumption.
    % So we define utility function of IaaS provider $i$ as follow:
    \begin{equation}
      \varphi_i(p_i, U_i, \Lambda_S^i)=\sum_{s=1}^{N_S}{p_i \lambda_{i,s}^i R_s} - P^{electrical}(P_i(\sum_{s=1}^{N_S}(u_{i,s})))
    \end{equation}


    \begin{equation}
      P_i(u_i) = P_i^{idle} + (P_i^{max} - P_i^{idle}) u_i
    \end{equation}
    So each IaaS provider try to solve following optimizaiton problem:
    \begin{subequations}
      \begin{align}
        &\max_{p_i, U_i,\Lambda^i_S} \varphi_i(p_i, U_i, \Lambda^i_S)\\
        &\text{subject to:} \nonumber\\
        &0 \le u_s^i, \forall s \in \{1, \hdots, N_S\} \\
        &\sum_{s=1}^{N_S}u_s^i \le 1 \\
        &P_{idle}^i + (P_{max}^i - P_{idle}^i)\sum_{s=1}^{N_S}u_s^i \le \bar{P^i} \\
        &\Lambda_s^i \in SOL(F_s), \forall s
      \end{align}
    \end{subequations}
    % We can define an exact potential function for this game.
    % As explained in ** function $\pi$ is an exact potential function if for --- we have:
    \begin{equation}
      \pi(x^i,x^{-i}) - \pi(y^i,x^{-i}) = \varphi_i(x^i,x^{-i}) - \varphi_i(y^i,x^{-i})
    \end{equation}
    It's easy to show that any global minimum of function $\pi$ is a nash equilibrium of corresponding game.

    show or not???
    For IaaS providers exact potential function can be writen as:
    \begin{equation}
      \pi(x^i, x^{-i}) = \sum_{i=1}^{N_I} \varphi_i(x^i, x^{-i})
    \end{equation}
  \section{Problem Formulation}
    \begin{subequations}
      \begin{align}
        &\max_{p^i, U^i,\Lambda^i_S, \sigma_S^i, \gamma_S^i, \nu_S^i,, \eta_S^i} \varphi(p_i, U^i,\Lambda_S^i)\\
        &\text{subject to:} \nonumber\\
        &\sum_{s=1}^{N_S}u_s^i \le 1 \\
        &0 \le u_s^i, \forall s \in \{1, \hdots, N_S\} \\
        &P_{idle}^i + (P_{max}^i - P_{idle}^i)\sum_{s=1}^{N_S}u_s^i \le \bar{P^i} \\
        &0 \le \lambda_{s,i}^i \\
        &\lambda_s^{min} - \sum_{j=1}^{N_{I}} \lambda_{j,s}^i \le 0, \forall s \in \{1, \hdots, N_S\} \\
        &\lambda_{j,s}^i R_s \le 0.9 \mu_{j,s} C_j, \forall s \& j  \\
        \begin{split}
          \sum_{j=1}^{N_I} \lambda_{j,s}^i  (t^{sensors}_{j,s} + \frac{1}{\mu_{j,s}^i - \lambda_{j,s}^i} + t_{j,s}^{actuators} - t_s^{\text{max}}) \\
          \le 0, \forall s
        \end{split} \\
        &\sigma_{j,s}^i \lambda_{j,s}^i = 0, \forall j \& s\\
        &\gamma_s^i (\lambda_s^{min} - \sum_{j=1}^{N_{I}} \lambda_{j,s}^i) = 0, \forall s\\
        &\nu_{j,s}^i (\lambda_{j,s} R_s - 0.9 \mu_{j,s} C_j) = 0 \forall j \& s\\
        \begin{split}
          \eta_s^i \sum_{j=1}^{N_I} \lambda_{j,s}^i  (t^{sensors}_{j,s} + \frac{1}{\mu_{j,s} C_j - \lambda_{j,s} R_s} + t_{j,s}^{actuators} - t_s^{\text{max}}) \\
          = 0, \forall s
        \end{split} \\
        \begin{split}
          &\frac{1}{\sum_{k=1}^{N_I} \lambda_{k,s}^i} - p_j R_s + \\
          &\sigma_{j,s}^i + \gamma_s^i - R_s \nu_{j,s}^i - \\
          &\eta_s^i (t_{j,s}^{sensors} + t_{j,s}^{actuators} - t_s^{max} + \frac{\mu_{j,s} C_j}{\mu_{j,s} C_j - \lambda_{j,s}^i R_s}) \\
          & = 0, \forall s \& j
        \end{split} \\
        &0 \le \sigma_{j,s}^i, \sigma_{j,s}^i, \nu_{j,s}^i \& \sigma_{j,s}^i, \forall s \& j
      \end{align}
    \end{subequations}
    Here $\lambda_{s,i}$ is $[\lambda^i_1, \cdots, \lambda^i_{N_S}]$ and $\lambda^i_s$ is conjecture of $\lambda_s$ by PaaS provider $i$, $u^i$ is $[u_{i,1}, \cdots, u_{i,N_S}]$ and $u_i = \sum_{i=1}^{N_S}u_{i,s}$.
    A potential function can be defined for this game.

    Function $\pi$ is potential function of this game:
    \begin{equation}
      \pi(x^i, y^i,x^{-i}, y^{-i})=\sum_{i=1}^{N_I} P(u_i) - p_i u_i
    \end{equation}
    here $x^i$ and $y^i$ are a tuple of $(p^i,u^i)$ and  $(\lambda_{s,i},\nu_S^i,\gamma_S^i)$

    \bibliographystyle{IEEEtran}
    \bibliography{references}

\end{document}
