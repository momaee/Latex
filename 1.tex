\documentclass[conference]{IEEEtran}
\IEEEoverridecommandlockouts
% The preceding line is only needed to identify funding in the first footnote. If that is unneeded, please comment it out.
\usepackage{cite}
\usepackage{amsmath,amssymb,amsfonts}
\usepackage{algorithmic}
\usepackage{graphicx}
\usepackage{textcomp}
\usepackage{algorithm}
\def\BibTeX{{\rm B\kern-.05em{\sc i\kern-.025em b}\kern-.08em
  T\kern-.1667em\lower.7ex\hbox{E}\kern-.125emX}}
\begin{document}

\title{Paper Title*}

\author{
  \IEEEauthorblockN{1\textsuperscript{st} Given Name Surname}
  \IEEEauthorblockA{\textit{dept. name of organization (of Aff.)} \\
  \textit{name of organization (of Aff.)}\\
  City, Country \\
  email address}
\and
  \IEEEauthorblockN{2\textsuperscript{nd} Given Name Surname}
  \IEEEauthorblockA{\textit{dept. name of organization (of Aff.)} \\
  \textit{name of organization (of Aff.)}\\
  City, Country \\
  email address}
}

\maketitle

\begin{abstract}

\end{abstract}

\begin{IEEEkeywords}

\end{IEEEkeywords}

  \section{Introduction}
  \section{System Model}
    % \cite{ref1}

    % \begin{enumerate}
      % \item
      % \item
    % \end{enumerate}
    % We consider that measuremnts of sensors for service $s$ are performed and sent to processing units by rate of $\sum_{i=1}^{N_I} r_{i,s}$ and $r_{i,s}$ is sampeling rate processed by IaaS provider $i$ for service $s$, hence we use $\log(\sum_{i=1}^{N_I} r_{i,s})$ as a quality of experience measurement for user of service $s$.
    % Let $p_i$ be the price that should be paid to IaaS provider $i$ for unit of processing power and $F_s$ be the processing power required for each sensor sample of user $s$.
    % Then $p_i r_{i,s} F_s$ is price paid to IaaS provider $i$ by user of service $s$ and total cost for user of service $s$ is $\sum_{i=1}^{N_I}{p_i r_{i,s} F_s}$.
    % We define utility function for user of service $s$ as the difference between quality of experience indicator and his total cost:
    \subsection{Graph Model}

    $V^c$ is set of cloud nodes.
    $V^f$ is set of fog nodes.
    $V^e$ is set of edge nodes.
    $V^s$ is set of sensor nodes.
    \begin{subequations}
      \begin{align}
          V^c = \{v_1^c, v_2^c, ..., v_{l_c}^c\} => |V^c| = l_c\\
          V^f = \{v_1^f, v_2^f, ..., v_{l_f}^f\} => |V^f| = l_f\\
          V^e = \{v_1^e, v_2^e, ..., v_{l_e}^e\} => |V^e| = l_e\\
          V^s = \{v_1^s, v_2^s, ..., v_{l_s}^s\} => |V^s| = l_s
      \end{align}
    \end{subequations}

    $C^c$ is set of capacity of cloud nodes.
    $C^f$ is set of capacity of fog nodes.
    $C^e$ is set of capacity of edge nodes.
    \begin{subequations}
      \begin{align}
          C^c = \{c_1^c, c_2^c, ..., c_{l_c}^c\}\\
          C^f = \{c_1^f, c_2^f, ..., c_{l_f}^f\}\\
          C^e = \{c_1^e, c_2^e, ..., c_{l_e}^e\}
      \end{align}
    \end{subequations}

    $T$ is set of tasks.
    \begin{subequations}
      \begin{align}
          T = \{t_1, t_2, ..., t_{l_t}\} => |T| = l_t\\
      \end{align}
    \end{subequations}
    Each task expresses as follows:
    \begin{equation}
      t_k \in T => t_k = (w_k, \delta_k, o_k)
    \end{equation}
    $w_k$ shows computation workload of task.
    $\delta_k$ is completion deadline of task and
    $o_k$ determines the owner of task.\\


    Price unit sets are defined as following:
    \begin{subequations}
      \begin{align}
        P^c = \{p_1^c, p_2^c, ..., p_{l_c}^c\}\\
        P^f = \{p_1^f, p_2^f, ..., p_{l_f}^f\}\\
        P^e = \{p_1^e, p_2^e, ..., p_{l_e}^e\}
      \end{align}
    \end{subequations}
    $P^c$, $P^f$ and $P^e$ are unit price set of using cloud nodes,
    fog nodes and edge nodes respectively.

    Transmition delay sets that show required time for trasmiting packets
    from sensors to each computational node are defined as follows:
    \begin{subequations}
      \begin{align}
        T^{c,tr} = \{\tau_1^{c,tr}, \tau_2^{c,tr}, ..., \tau_{l_c}^{c,tr}\}\\
        T^{f,tr} = \{\tau_1^{f,tr}, \tau_2^{f,tr}, ..., \tau_{l_f}^{f,tr}\}\\
        T^{e,tr} = \{\tau_1^{e,tr}, \tau_2^{e,tr}, ..., \tau_{l_e}^{e,tr}\}
      \end{align}
    \end{subequations}

    We define execution rate for each computational node as follows:
    \begin{subequations}
      \begin{align}
        R^c = \{r_1^c, r_2^c, ..., r_{l_c}^c\}\\
        R^f = \{r_1^f, r_2^f, ..., r_{l_f}^f\}\\
        R^e = \{r_1^e, r_2^e, ..., r_{l_e}^e\}
      \end{align}
    \end{subequations}

    \subsection{Variables}

    We define three integer variables for allocating tasks between nodes.
    \begin{subequations}
      \begin{align}
      x_{k,h}^c =
      \begin{cases}
        1 & \text{task $k$ is allocated to cloud node $h$} \\
        0 & \text{o.w.}
     \end{cases}
     \end{align}

     \begin{align}
       x_{k,j}^f =
       \begin{cases}
         1 & \text{task $k$ is allocated to fog node $j$} \\
         0 & \text{o.w.}
      \end{cases}
     \end{align}

     \begin{align}
       x_{k,i}^e =
       \begin{cases}
         1 & \text{task $k$ is allocated to edge node $i$} \\
         0 & \text{o.w.}
      \end{cases}
     \end{align}

     \begin{align}
       \tau_k = \sum_{i=1}^{l_e}x_{k,i}^e(\tau_i^{e,tr}+\frac{w_k}{r_i^e}) \\ \notag
       +\sum_{j=1}^{l_f}x_{k,j}^f(\tau_j^{f,tr}+\frac{w_k}{r_j^f}) \\ \notag
       +\sum_{h=1}^{l_c}x_{k,h}^c(\tau_k^{c,tr}+\frac{w_k}{r_h^c})
     \end{align}
    \end{subequations}

    \subsection{Constraints}
    \begin{subequations}
      \begin{align}
        \tau_k \le \delta_k \quad\quad \forall k\in\{1,2,...,l_t\}
      \end{align}
      \begin{align}
        \sum_{k=1}^{l_t}x_{k,h}^c w_k \le c_h^c \quad\quad \forall h\in\{1,2,...,l_c\}
      \end{align}
      \begin{align}
        \sum_{k=1}^{l_t}x_{k,j}^f w_k \le c_j^f \quad\quad \forall j\in\{1,2,...,l_f\}
      \end{align}
      \begin{align}
        \sum_{k=1}^{l_t}x_{k,i}^e w_k \le c_i^e \quad\quad \forall i\in\{1,2,...,l_e\}
      \end{align}
      \begin{align}
        \sum_{i=1}^{l_e}x_{k,i}^e+\sum_{j=1}^{l_f}x_{k,j}^f+\sum_{h=1}^{l_c}x_{k,h}^c=1 \quad\quad \forall k\in\{1,2,...,l_t\}
      \end{align}
    \end{subequations}

    \subsection{Objective}

    \begin{subequations}
        \begin{align}
          p_k = \sum_{i=1}^{l_e}x_{k,i}^e C(v_i^e, t_k) \\ \notag
          +\sum_{j=1}^{l_f}x_{k,j}^f C(v_j^f, t_k) \\ \notag
          +\sum_{h=1}^{l_c}x_{k,h}^c C(v_h^c, t_k)
        \end{align}
    \end{subequations}
    \begin{subequations}
    \begin{align}
      \min{\sum_{k=1}^{l_t}{p_k}} \\ \notag
      \text{subject to: 9}
    \end{align}
    \end{subequations}

    \subsection{Solution}
    We can reshape main problem as following:
    \begin{subequations}
      \begin{align}
        &\min(\sum_{i=1}^{l_e} \sum_{k=1}^{l_t} x_{k,i}^e C_{k,i}^e \\ \notag
        &+\sum_{j=1}^{l_f} \sum_{k=1}^{l_t} x_{k,j}^f C_{k,j}^f
        +\sum_{h=1}^{l_c} \sum_{k=1}^{l_t} x_{k,h}^c C_{k,h}^c) \\ \notag
        &\text{subject to:} \\ \notag
          %
        &\sum_{i=1}^{l_e}x_{k,i}^e\tau_{k,i}^e
        +\sum_{j=1}^{l_f}x_{k,j}^f\tau_{k,j}^f
        +\sum_{h=1}^{l_c}x_{k,h}^c\tau_{k,h}^c
        \le \delta_k \quad \forall k\in\{1,...,l_t\} \\ \notag
          %
        &\sum_{k=1}^{l_t}x_{k,h}^c w_k \le c_h^c \quad\quad \forall h\in\{1,2,...,l_c\} \\ \notag
          %
        &\sum_{k=1}^{l_t}x_{k,j}^f w_k \le c_j^f \quad\quad \forall j\in\{1,2,...,l_f\} \\ \notag
          %
        &\sum_{k=1}^{l_t}x_{k,i}^e w_k \le c_i^e \quad\quad \forall i\in\{1,2,...,l_e\} \\ \notag
          %
        &\sum_{i=1}^{l_e}x_{k,i}^e+\sum_{j=1}^{l_f}x_{k,j}^f+\sum_{h=1}^{l_c}x_{k,h}^c=1 \quad\quad
        \forall k\in\{1,2,...,l_t\} \\ \notag
      \end{align}
    \end{subequations}

    We define $u^m$ for each computational agent $m$, that is a matrix with size $l_t*(l_e+l_f+l_c)$.
    It is the local copy of all variables in agent $m$. i.e. $u_{k,i}^{e,m}$ is the copy of variable $x_{k,i}^e$
    in agent $m$ for $m=1,..,(l_m = l_e+l_f+l_c)$. So we should add new constraint $u^m = z \quad \forall m$ to main problem.
    We will use admm on this new constraint so:

    \begin{subequations}
      \begin{align}
        &L_p = \sum_{i=1}^{l_e} \sum_{k=1}^{l_t} x_{k,i}^e C_{k,i}^e
        +\sum_{j=1}^{l_f} \sum_{k=1}^{l_t} x_{k,j}^f C_{k,j}^f
        +\sum_{h=1}^{l_c} \sum_{k=1}^{l_t} x_{k,h}^c C_{k,h}^c \\ \notag
        &+\sum_{m=1}^{l_m} \nu^m*(u^m-z) + \sum_{m=1}^{l_m} \norm\frac{\rho}{2} \|u^m-z\|^2
      \end{align}
    \end{subequations}

    We can seperate augmented lagrangian for each computational agent $m$ then:
    \begin{subequations}
      \begin{align}
        &L_p^m = \sum_{k=1}^{l_t} u_{k,m}^{m} C_{k,m}
        + \nu^m*(u^m-z) + \norm\frac{\rho}{2} \|u^m-z\|^2 \\ \notag
        &\quad \forall m\in\{1,2,...,l_m\}
      \end{align}
    \end{subequations}

    So we can write the algorithm as following:
    \begin{subequations}
      \begin{align}
        &\text{for each iteration k} \\ \notag
        &\text{1.\quad}u^{m,(k+1)} = arg \min {L_p^m(u^{m}, z^{(k)}, \nu^{m,(k)})} = \\ \notag
        &\sum_{k=1}^{l_t} u_{k,m}^{m} C_{k,m}
        + \nu^{m,(k)}*(u^m-z^{(k)}) + \norm\frac{\rho}{2} \|u^m-z^{(k)}\|^2 \\ \notag
        &\text{subject to:} \\ \notag
        &\sum_{k=1}^{l_t}{u_{k,m}^m w_k} \le c^m \\ \notag
        &\sum_{i=1}^{l_e}u_{k,i}^{e,m}+\sum_{j=1}^{l_f}u_{k,j}^{f,m}+\sum_{h=1}^{l_c}u_{k,h}^{c,m}=1 \quad\quad \forall k\in\{1,2,...,l_t\} \\ \notag
        &\sum_{i=1}^{l_e}u_{k,i}^{e,m}\tau_{k,i}^e
        +\sum_{j=1}^{l_f}u_{k,j}^{f,m}\tau_{k,j}^f
        +\sum_{h=1}^{l_c}u_{k,h}^{c,m}\tau_{k,h}^c
        \le \delta_k \quad\quad \forall k\in\{1,2,...,l_t\} \\ \notag
        &\text{2.\quad}{z}^{(k+1)} = \bar{u}^{(k+1)} + \frac{1}{\rho}\bar{\nu}^{(k)} \\ \notag
        &\text{3.\quad}{\nu}^{m,(k+1)} = \nu^{m,(k)} + \rho(u^{m,(k+1)} - z^{(k+1)})
      \end{align}
    \end{subequations}

    \subsection{Solution 2}
    lagrangian of main problem is as following
    \begin{subequations}
      \begin{align}
        &L(x^e,x^f,x^c,\lambda,\nu) = \sum_{i=1}^{l_e} \sum_{k=1}^{l_t} x_{k,i}^e C_{k,i}^e
        +\sum_{j=1}^{l_f} \sum_{k=1}^{l_t} x_{k,j}^f C_{k,j}^f
        +\sum_{h=1}^{l_c} \sum_{k=1}^{l_t} x_{k,h}^c C_{k,h}^c \\ \notag
        %
        &+\sum_{k=1}^{l_t}\lambda_k(\sum_{i=1}^{l_e}x_{k,i}^e\tau_{k,i}^e
        +\sum_{j=1}^{l_f}x_{k,j}^f\tau_{k,j}^f
        +\sum_{h=1}^{l_c}x_{k,h}^c\tau_{k,h}^c
        -\delta_k) \\ \notag
        %
        &+\sum_{k=1}^{l_t}\nu_k(\sum_{i=1}^{l_e}x_{k,i}^e+\sum_{j=1}^{l_f}x_{k,j}^f+\sum_{h=1}^{l_c}x_{k,h}^c-1)
      \end{align}
    \end{subequations}
    So we can decompose the lagrangian as follows
    \begin{subequations}
      \begin{align}
        &L(x^e,x^f,x^c,\lambda,\nu) = \\ \notag
        &\sum_{i=1}^{l_e} \sum_{k=1}^{l_t} (x_{k,i}^e C_{k,i}^e +\lambda_kx_{k,i}^e\tau_{k,i}^e +\nu_kx_{k,i}^e - \frac{\lambda_k\delta_k +\nu_k}{3l_e} ) \\ \notag
        &+\sum_{j=1}^{l_f} \sum_{k=1}^{l_t} (x_{k,j}^f C_{k,j}^f +\lambda_kx_{k,j}^f\tau_{k,j}^f +\nu_kx_{k,j}^f - \frac{\lambda_k\delta_k +\nu_k}{3l_f}) \\ \notag
        &+\sum_{h=1}^{l_c} \sum_{k=1}^{l_t} (x_{k,h}^c C_{k,h}^c +\lambda_kx_{k,h}^c\tau_{k,h}^c +\nu_kx_{k,h}^c - \frac{\lambda_k\delta_k +\nu_k}{3l_c}) \\ \notag
      \end{align}
    \end{subequations}
    \begin{subequations}
      \begin{align}
        &L(x^e,x^f,x^c,\lambda,\nu) = \sum_{i=1}^{l_e} L_{i}^e(x_{i}^e,\lambda,\nu) \\ \notag
        &+\sum_{j=1}^{l_f} L_{j}^f(x_{j}^f,\lambda,\nu) \\ \notag
        &+\sum_{h=1}^{l_c} L_{h}^c(x_{h}^c,\lambda,\nu) \\ \notag
      \end{align}
    \end{subequations}
    \begin{subequations}
      \begin{align}
        &g(\lambda, \nu) = \inf_{x^e,x^f,x^c}L(x^e,x^f,x^c,\lambda,\nu) \\ \notag
        &=\sum_{i=1}^{l_e}\inf_{x_i^e}L_{i}^e(x_{i}^e,\lambda,\nu) \\ \notag
        &+\sum_{j=1}^{l_f}\inf_{x_j^f}L_{j}^f(x_{j}^f,\lambda,\nu) \\ \notag
        &+\sum_{h=1}^{l_c}\inf_{x_h^c}L_{h}^c(x_{h}^c,\lambda,\nu) \\ \notag
        &=\sum_{i=1}^{l_e}g_{i}^e(\lambda,\nu) \\ \notag
        &+\sum_{j=1}^{l_f}g_{j}^f(\lambda,\nu) \\ \notag
        &+\sum_{h=1}^{l_c}g_{h}^c(\lambda,\nu) \\ \notag
      \end{align}
    \end{subequations}

    \begin{subequations}
      \begin{align}
        &\lambda_k^+ = \lambda_k^- + \alpha(\sum_{i=1}^{l_e}x_{k,i}^e\tau_{k,i}^e
        +\sum_{j=1}^{l_f}x_{k,j}^f\tau_{k,j}^f
        +\sum_{h=1}^{l_c}x_{k,h}^c\tau_{k,h}^c
        -\delta_k) \\ \notag
      \end{align}
      \begin{align}
        &\nu_k^+ = \nu_k^- + \alpha(\sum_{i=1}^{l_e}x_{k,i}^e+\sum_{j=1}^{l_f}x_{k,j}^f+\sum_{h=1}^{l_c}x_{k,h}^c-1)
      \end{align}
    \end{subequations}

    \begin{algorithm}
      \caption{}
      \begin{algorithmic}
        \WHILE{not converged}
        1. update $x^c, x^f, x^e$ using (19a) \newline
        2. update lagrangian multipliers using (20a) and (21a) \newline
        $\alpha = \frac{\alpha_0}{k}$ for each iteration $k$
        \ENDWHILE
      \end{algorithmic}
    \end{algorithm}


    % \begin{equation}
    %   P_i(u_i) = P_i^{idle} + (P_i^{max} - P_i^{idle}) u_i
    % \end{equation}
    % So each IaaS provider try to solve following optimizaiton problem:
    % \begin{subequations}
    %   \begin{align}
    %     &\max_{p_i, U_i,\Lambda^i_S} \varphi_i(p_i, U_i, \Lambda^i_S)\\
    %     &\text{subject to:} \nonumber\\
    %     &0 \le u_s^i, \forall s \in \{1, \hdots, N_S\} \\
    %     &\sum_{s=1}^{N_S}u_s^i \le 1 \\
    %     &P_{idle}^i + (P_{max}^i - P_{idle}^i)\sum_{s=1}^{N_S}u_s^i \le \bar{P^i} \\
    %     &\Lambda_s^i \in SOL(F_s), \forall s
    %   \end{align}
    % \end{subequations}




    % We can define an exact potential function for this game.
    % As explained in ** function $\pi$ is an exact potential function if for --- we have:




  %
  %   \begin{equation}
  %     \pi(x^i,x^{-i}) - \pi(y^i,x^{-i}) = \varphi_i(x^i,x^{-i}) - \varphi_i(y^i,x^{-i})
  %   \end{equation}
  %   It's easy to show that any global minimum of function $\pi$ is a nash equilibrium of corresponding game.
  %
  %   show or not???
  %   For IaaS providers exact potential function can be writen as:
  %   \begin{equation}
  %     \pi(x^i, x^{-i}) = \sum_{i=1}^{N_I} \varphi_i(x^i, x^{-i})
  %   \end{equation}
  % \section{Problem Formulation}
  %   \begin{subequations}
  %     \begin{align}
  %       &\max_{p^i, U^i,\Lambda^i_S, \sigma_S^i, \gamma_S^i, \nu_S^i,, \eta_S^i} \varphi(p_i, U^i,\Lambda_S^i)\\
  %       &\text{subject to:} \nonumber\\
  %       &\sum_{s=1}^{N_S}u_s^i \le 1 \\
  %       &0 \le u_s^i, \forall s \in \{1, \hdots, N_S\} \\
  %       &P_{idle}^i + (P_{max}^i - P_{idle}^i)\sum_{s=1}^{N_S}u_s^i \le \bar{P^i} \\
  %       &0 \le \lambda_{s,i}^i \\
  %       &\lambda_s^{min} - \sum_{j=1}^{N_{I}} \lambda_{j,s}^i \le 0, \forall s \in \{1, \hdots, N_S\} \\
  %       &\lambda_{j,s}^i R_s \le 0.9 \mu_{j,s} C_j, \forall s \& j  \\
  %       \begin{split}
  %         \sum_{j=1}^{N_I} \lambda_{j,s}^i  (t^{sensors}_{j,s} + \frac{1}{\mu_{j,s}^i - \lambda_{j,s}^i} + t_{j,s}^{actuators} - t_s^{\text{max}}) \\
  %         \le 0, \forall s
  %       \end{split} \\
  %       &\sigma_{j,s}^i \lambda_{j,s}^i = 0, \forall j \& s\\
  %       &\gamma_s^i (\lambda_s^{min} - \sum_{j=1}^{N_{I}} \lambda_{j,s}^i) = 0, \forall s\\
  %       &\nu_{j,s}^i (\lambda_{j,s} R_s - 0.9 \mu_{j,s} C_j) = 0 \forall j \& s\\
  %       \begin{split}
  %         \eta_s^i \sum_{j=1}^{N_I} \lambda_{j,s}^i  (t^{sensors}_{j,s} + \frac{1}{\mu_{j,s} C_j - \lambda_{j,s} R_s} + t_{j,s}^{actuators} - t_s^{\text{max}}) \\
  %         = 0, \forall s
  %       \end{split} \\
  %       \begin{split}
  %         &\frac{1}{\sum_{k=1}^{N_I} \lambda_{k,s}^i} - p_j R_s + \\
  %         &\sigma_{j,s}^i + \gamma_s^i - R_s \nu_{j,s}^i - \\
  %         &\eta_s^i (t_{j,s}^{sensors} + t_{j,s}^{actuators} - t_s^{max} + \frac{\mu_{j,s} C_j}{\mu_{j,s} C_j - \lambda_{j,s}^i R_s}) \\
  %         & = 0, \forall s \& j
  %       \end{split} \\
  %       &0 \le \sigma_{j,s}^i, \sigma_{j,s}^i, \nu_{j,s}^i \& \sigma_{j,s}^i, \forall s \& j
  %     \end{align}
  %   \end{subequations}
  %   Here $\lambda_{s,i}$ is $[\lambda^i_1, \cdots, \lambda^i_{N_S}]$ and $\lambda^i_s$ is conjecture of $\lambda_s$ by PaaS provider $i$, $u^i$ is $[u_{i,1}, \cdots, u_{i,N_S}]$ and $u_i = \sum_{i=1}^{N_S}u_{i,s}$.
  %   A potential function can be defined for this game.
  %
  %   Function $\pi$ is potential function of this game:
  %   \begin{equation}
  %     \pi(x^i, y^i,x^{-i}, y^{-i})=\sum_{i=1}^{N_I} P(u_i) - p_i u_i
  %   \end{equation}
  %   here $x^i$ and $y^i$ are a tuple of $(p^i,u^i)$ and  $(\lambda_{s,i},\nu_S^i,\gamma_S^i)$







    \bibliographystyle{IEEEtran}
    \bibliography{references}

\end{document}
